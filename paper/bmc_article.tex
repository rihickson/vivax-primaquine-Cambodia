%% BioMed_Central_Tex_Template_v1.06
%%                                      %
%  bmc_article.tex            ver: 1.06 %
%                                       %

%%IMPORTANT: do not delete the first line of this template
%%It must be present to enable the BMC Submission system to
%%recognise this template!!

%%%%%%%%%%%%%%%%%%%%%%%%%%%%%%%%%%%%%%%%%
%%                                     %%
%%  LaTeX template for BioMed Central  %%
%%     journal article submissions     %%
%%                                     %%
%%          <8 June 2012>              %%
%%                                     %%
%%                                     %%
%%%%%%%%%%%%%%%%%%%%%%%%%%%%%%%%%%%%%%%%%


%%%%%%%%%%%%%%%%%%%%%%%%%%%%%%%%%%%%%%%%%%%%%%%%%%%%%%%%%%%%%%%%%%%%%
%%                                                                 %%
%% For instructions on how to fill out this Tex template           %%
%% document please refer to Readme.html and the instructions for   %%
%% authors page on the biomed central website                      %%
%% http://www.biomedcentral.com/info/authors/                      %%
%%                                                                 %%
%% Please do not use \input{...} to include other tex files.       %%
%% Submit your LaTeX manuscript as one .tex document.              %%
%%                                                                 %%
%% All additional figures and files should be attached             %%
%% separately and not embedded in the \TeX\ document itself.       %%
%%                                                                 %%
%% BioMed Central currently use the MikTex distribution of         %%
%% TeX for Windows) of TeX and LaTeX.  This is available from      %%
%% http://www.miktex.org                                           %%
%%                                                                 %%
%%%%%%%%%%%%%%%%%%%%%%%%%%%%%%%%%%%%%%%%%%%%%%%%%%%%%%%%%%%%%%%%%%%%%

%%% additional documentclass options:
%  [doublespacing]
%  [linenumbers]   - put the line numbers on margins

%%% loading packages, author definitions

%\documentclass[twocolumn]{bmcart}% uncomment this for twocolumn layout and comment line below
\documentclass[doublespacing]{bmcart}

%%% Load packages
\usepackage{amsthm,amsmath}
%\RequirePackage{natbib}
%\RequirePackage[authoryear]{natbib}% uncomment this for author-year bibliography
\RequirePackage{hyperref}
%\usepackage[utf8]{inputenc} %unicode support
\usepackage[applemac]{inputenc} %applemac support if unicode package fails
%\usepackage[latin1]{inputenc} %UNIX support if unicode package fails


%%%%%%%%%%%%%%%%%%%%%%%%%%%%%%%%%%%%%%%%%%%%%%%%%
%%                                             %%
%%  If you wish to display your graphics for   %%
%%  your own use using includegraphic or       %%
%%  includegraphics, then comment out the      %%
%%  following two lines of code.               %%
%%  NB: These line *must* be included when     %%
%%  submitting to BMC.                         %%
%%  All figure files must be submitted as      %%
%%  separate graphics through the BMC          %%
%%  submission process, not included in the    %%
%%  submitted article.                         %%
%%                                             %%
%%%%%%%%%%%%%%%%%%%%%%%%%%%%%%%%%%%%%%%%%%%%%%%%%


\def\includegraphic{}
\def\includegraphics{}


\usepackage{todonotes}
\usepackage{multirow}
%%% Put your definitions there:
\startlocaldefs
\newcommand{\pv}{\textit{P. vivax}}
\newcommand{\pf}{\textit{P. falciparum}}
\endlocaldefs


%%% Begin ...
\begin{document}

%%% Start of article front matter
\begin{frontmatter}

\begin{fmbox}
\dochead{Research}

%%%%%%%%%%%%%%%%%%%%%%%%%%%%%%%%%%%%%%%%%%%%%%
%%                                          %%
%% Enter the title of your article here     %%
%%                                          %%
%%%%%%%%%%%%%%%%%%%%%%%%%%%%%%%%%%%%%%%%%%%%%%

\title{Assessing a primaquine intervention in Cambodia 2020--2025 to control vivax malaria}

%%%%%%%%%%%%%%%%%%%%%%%%%%%%%%%%%%%%%%%%%%%%%%
%%                                          %%
%% Enter the authors here                   %%
%%                                          %%
%% Specify information, if available,       %%
%% in the form:                             %%
%%   <key>={<id1>,<id2>}                    %%
%%   <key>=                                 %%
%% Comment or delete the keys which are     %%
%% not used. Repeat \author command as much %%
%% as required.                             %%
%%                                          %%
%%%%%%%%%%%%%%%%%%%%%%%%%%%%%%%%%%%%%%%%%%%%%%

%\author[
%   addressref={aff1},                   % id's of addresses, e.g. {aff1,aff2}
%   corref={aff1},                       % id of corresponding address, if any
%   noteref={n1},                        % id's of article notes, if any
%   email={jane.e.doe@cambridge.co.uk}   % email address
%]{\inits{JE}\fnm{Jane E} \snm{Doe}}
%\author[
%   addressref={aff1,aff2},
%   email={john.RS.Smith@cambridge.co.uk}
%]{\inits{JRS}\fnm{John RS} \snm{Smith}}
\author[
   addressref={melbms,melbph,bi},                   % id's of addresses, e.g. {aff1,aff2}
   %corref={aff1},                       % id of corresponding address, if any
   %noteref={n1},                        % id's of article notes, if any
   email={r.hickson@UNSWalumni.com}   % email address
]{Order to be confirmed: \inits{RI}\fnm{RI} \snm{Hickson}}
%
\author[
   addressref={bi},
   email={rowan.martin-hughes@burnet.edu.au}
]{\inits{R}\fnm{Rowan} \snm{Martin-Hughes}}
%
\author[
   addressref={menzies},
   email={angela.devine@menzies.edu.au}
]{\inits{A}\fnm{Angela} \snm{Devine}}
%
\author[
   addressref={melbph,doherty},
   email={david.price1@unimelb.edu.au}
]{\inits{DJ}\fnm{David J} \snm{Price}}
% from here, excepting senior, authors are alphabetical
\author[
   addressref={bi},
   email={freya.fowkes@burnet.edu.au}
]{\inits{FJI}\fnm{Freya JI} \snm{Fowkes}}
%
\author[
   addressref={melbms,melbph,doherty},
   email={jamesm@unimelb.edu.au}
]{\inits{JM}\fnm{James M} \snm{McCaw}}
%
\author[
   addressref={menzies},
   email={@menzies.edu.au}
]{(TBC: \inits{R}\fnm{Ric} \snm{Price})}
%
\author[
   addressref={melbph},
   email={julieas@unimelb.edu.au}
]{\inits{JA}\fnm{Julie A} \snm{Simpson}}
%
\author[
   addressref={cnm},
   email={sivsovannaroths@gmail.com}
]{\inits{S}\fnm{Siv} \snm{Sovannaroth}}
%
\author[
   addressref={cnm,moru},
   email={pengby@email.com}
]{\inits{P}\fnm{Pengby} \snm{Ngor}}

%%%%%%%%%%%%%%%%%%%%%%%%%%%%%%%%%%%%%%%%%%%%%%
%%                                          %%
%% Enter the authors' addresses here        %%
%%                                          %%
%% Repeat \address commands as much as      %%
%% required.                                %%
%%                                          %%
%%%%%%%%%%%%%%%%%%%%%%%%%%%%%%%%%%%%%%%%%%%%%%

%\address[id=aff1]{%                           % unique id
%  \orgname{Department of Zoology, Cambridge}, % university, etc
%  %\street{Waterloo Road},                     %
%  %\postcode{}                                % post or zip code
%  \city{London},                              % city
%  \cny{UK}                                    % country
%}
%\address[id=aff2]{%
%  \orgname{Marine Ecology Department, Institute of Marine Sciences Kiel},
%  %\street{D\"{u}sternbrooker Weg 20},
%  %\postcode{24105}
%  \city{Kiel},
%  \cny{Germany}
%}
\address[id=melbms]{%                           % unique id
  \orgname{School of Mathematics and Statistics, Faculty of Science, University of Melbourne}, % university, etc
  %\street{},                     %
  %\postcode{}                                % post or zip code
  \city{Parkville},                              % city
  \cny{Australia}                                    % country
}
\address[id=melbph]{%
  \orgname{Melbourne School of Population and Global Health, Faculty of Medicine, Dentistry, and Health Sciences, University of Melbourne}, % university, etc
  %\street{},                     %
  %\postcode{}                                % post or zip code
  \city{Parkville},                              % city
  \cny{Australia}                                    % country
}
\address[id=bi]{%
  \orgname{Burnet Institute},
  %\street{},
  %\postcode{24105}
  \city{Melbourne},
  \cny{Australia}
}
\address[id=menzies]{%
  \orgname{Menzies School of Health Research},
  %\street{},
  %\postcode{24105}
  \city{Melbourne},
  \cny{Australia}
}
\address[id=doherty]{%
  \orgname{Doherty Institute},
  %\street{},
  %\postcode{24105}
  \city{Melbourne},
  \cny{Australia}
}
\address[id=cnm]{%
  \orgname{Cambodian National Center for Parasitology, Entomology and Malaria Control},
  %\street{},
  %\postcode{24105}
  \city{Phnom Penh},
  \cny{Cambodia}
}
\address[id=moru]{%
  \orgname{Mahidol-Oxford Tropical Medicine Research Unit, Faculty of Tropical Medicine, Mahidol University},
  %\street{},
  %\postcode{24105}
  \city{Bangkok},
  \cny{Thailand}
}

%%%%%%%%%%%%%%%%%%%%%%%%%%%%%%%%%%%%%%%%%%%%%%
%%                                          %%
%% Enter short notes here                   %%
%%                                          %%
%% Short notes will be after addresses      %%
%% on first page.                           %%
%%                                          %%
%%%%%%%%%%%%%%%%%%%%%%%%%%%%%%%%%%%%%%%%%%%%%%

\begin{artnotes}
%\note{Sample of title note}     % note to the article
%\note[id=n1]{Equal contributor} % note, connected to author
\end{artnotes}

\end{fmbox}% comment this for two column layout

%%%%%%%%%%%%%%%%%%%%%%%%%%%%%%%%%%%%%%%%%%%%%%
%%                                          %%
%% The Abstract begins here                 %%
%%                                          %%
%% Please refer to the Instructions for     %%
%% authors on http://www.biomedcentral.com  %%
%% and include the section headings         %%
%% accordingly for your article type.       %%
%%                                          %%
%%%%%%%%%%%%%%%%%%%%%%%%%%%%%%%%%%%%%%%%%%%%%%

\begin{abstractbox}
% Instructions from: https://malariajournal.biomedcentral.com/submission-guidelines/preparing-your-manuscript/research-article
%Abstract
%The Abstract should not exceed 350 words. Please minimize the use of abbreviations and do not cite references in the abstract. Reports of randomized controlled trials should follow the CONSORT extension for abstracts. The abstract must include the following separate sections:
%
%Background: the context and purpose of the study
%Methods: how the study was performed and statistical tests used
%Results: the main findings
%Conclusions: brief summary and potential implications
%Trial registration: If your article reports the results of a health care intervention on human participants, it must be registered in an appropriate registry and the registration number and date of registration should be in stated in this section. If it was not registered prospectively (before enrollment of the first participant), you should include the words 'retrospectively registered'. See our editorial policies for more information on trial registration

\begin{abstract} % abstract

\parttitle{Background} %the context and purpose of the study
Elimination targets for \textit{Plasmodium vivax} are approaching, with the Cambodian target 2025. Quantitative tools can help determine if proposed new strategies will be sufficient to meet those targets.

\parttitle{Methods} %how the study was performed and statistical tests used
We calibrated the Optima malaria transmission model reported case data from 2011--2018 for six Provinces with different transmission levels. The model had two human populations: with males 15 years plus, and everyone else. We used the calibrated model to explore for best and worst case interpretations of the available case data, and of the Primaquine intervention.

\parttitle{Results} %the main findings
We found elimination is unlikely to be reached in Provinces with fairly high burdens of \textit{Plasmodium vivax}, such as Pursat, by only targeting adult males with Primaquine. However, it will substantially reduce transmission. As such, we identify how many tests will need to be conducted to have 99\% confidence of detecting at least one case, given the lower incidence by 2025.

\parttitle{Conclusions}
A primaquine intervention targeting adult males is likely to have a substantial impact on transmission of \pv, though it is not likely to result in elimination from all Provinces by the 2025 target. The surveillance requirements to ensure the resulting lower incidence is detected as Cambodia approaches elimination may be infeasible, e.g. for Takeo, especially as all Provinces will see a decrease in case counts as the intervention is Nationwide.

\end{abstract}

%%%%%%%%%%%%%%%%%%%%%%%%%%%%%%%%%%%%%%%%%%%%%%
%%                                          %%
%% The keywords begin here                  %%
%%                                          %%
%% Put each keyword in separate \kwd{}.     %%
%%                                          %%
%%%%%%%%%%%%%%%%%%%%%%%%%%%%%%%%%%%%%%%%%%%%%%

\begin{keyword}
\kwd{Malaria}
\kwd{\textit{Plasmodium vivax}}
\kwd{Transmission}
\kwd{Primaquine}
\kwd{Radical cure}
\kwd{Mathematical model}
\end{keyword}

% MSC classifications codes, if any
%\begin{keyword}[class=AMS]
%\kwd[Primary ]{}
%\kwd{}
%\kwd[; secondary ]{}
%\end{keyword}

\end{abstractbox}
%
%\end{fmbox}% uncomment this for twcolumn layout

\end{frontmatter}

%%%%%%%%%%%%%%%%%%%%%%%%%%%%%%%%%%%%%%%%%%%%%%
%%                                          %%
%% The Main Body begins here                %%
%%                                          %%
%% Please refer to the instructions for     %%
%% authors on:                              %%
%% http://www.biomedcentral.com/info/authors%%
%% and include the section headings         %%
%% accordingly for your article type.       %%
%%                                          %%
%% See the Results and Discussion section   %%
%% for details on how to create sub-sections%%
%%                                          %%
%% use \cite{...} to cite references        %%
%%  \cite{koon} and                         %%
%%  \cite{oreg,khar,zvai,xjon,schn,pond}    %%
%%  \nocite{smith,marg,hunn,advi,koha,mouse}%%
%%                                          %%
%%%%%%%%%%%%%%%%%%%%%%%%%%%%%%%%%%%%%%%%%%%%%%

%%%%%%%%%%%%%%%%%%%%%%%%% start of article main body
% <put your article body there>

%%%%%%%%%%%%%%%%
%% Background %%
%%
%Instructions for authors from: https://malariajournal.biomedcentral.com/submission-guidelines/preparing-your-manuscript/research-article
%Background
%The Background section should explain the background to the study, its aims, a summary of the existing literature and why this study was necessary or its contribution to the field.
\section*{Background} 

\textit{Plasmodium vivax} (\pv) is the cause of a significant burden of malaria globally, with an estimated XX cases, XX deaths~\cite{WHOreport}. In Cambodia, it has been responsible for 30--80\% of cases in different Provinces, with the proportion increasing as the burden of \textit{Plasmodium falciparum} (\pf) has decreased~\cite{Pengby}.

The key difference between \pf~and \pv~is the hypnozoite stage of \pv, which results in relapses~\cite{biologyofvivax}. There are an estimated XX~hypnozoites formed from each infectious mosquito bite, though the biology and mechanisms are poorly understood~\cite{vivaxbiology}. Standard treatment for \pv~is Chloroquine (CQ) for a blood stage infection. Radical cures have been developed to clear the hypnozoite stage, using 8-Aminoquinolines~\cite{RicPrice}. Primaquine (PQ) has been approved/licenses for use in several countries, though the WHO recommendation is to test for G6PDd before administration~\cite{WHOguidelines}. 

Elimination targets for \pv~have been set for many countries~\cite{WHO}, and the Cambodian target is 2025~\cite{Cambodia}. Cambodia are currently trialling a 14-day low dose primaquine intervention for adult males in a couple of health centres in Pursat Province. If successful, this will be expanded into a National programme. We use transmission modelling to determine if this is likely to be sufficient to eliminate \pv~by the 2025 target. 

%Methods
%The methods section should include:
%
%the aim, design and setting of the study
%the characteristics of participants or description of materials
%a clear description of all processes, interventions and comparisons. Generic drug names should generally be used. When proprietary brands are used in research, include the brand names in parentheses
%the type of statistical analysis used, including a power calculation if appropriate
\section*{Methods} \label{sec:methods} %Subsections are currently modelled on Scott et al. 2017

\subsection*{Data synthesis to assess disease burden} 
Data on malaria incidence were obtained from the Cambodia National Center for Parasitology, Entomology and Malaria Control (CNM)~\ref{CNM_Pengby}. Population and mortality data were obtained from the Cambodia Bureau of Statistics~\ref{}. The aggregated data used for model calibration are provided in Additional file~\ref{supp}, and on GitHub at \url{https://github.com/rihickson/vivax-primaquine-Cambodia}. \todo[inline]{Make this repo public.}  

\subsection*{Epidemic model}

The dynamic transmission model of \pv~is based on that by Scott~\etal~\cite{scott2017}. It accounts for transmission between humans and mosquitoes, with the conceptual disease progression in the model depicted in Fig~\ref{fig:model_flow}. People in the model are identified as: susceptible; infected with \pv~in the liver only (hypnozoites and/or active liver stage); infected with active blood stage infection able to infect mosquitoes (gametocytes present -- further divided into clinical, severe, or asymptomatic); or recovered and immune (with no hypnozoites). The recovered and immune compartment is important for capturing \pv~dynamics given our current understanding of the importance of immunity in reducing transmission or symptomatic infections~\cite{}\todo{fact check me}. The human population is stratified into males 15 years and older (M15+) and everyone else (Gen), to enable the primaquine intervention proposed in Cambodia to be captured by the model. Further details on the full model structure, parameter values, and calibration are provided in Additional file~\ref{supp} and on \url[Github]{https://github.com/rihickson/vivax-primaquine-Cambodia}.


\subsection*{Programmatic response considered}
For the sake of $<$fancy word meaning simplicity of model design$>$, the only explicit programmatic response considered is the primaquine intervention. The effect of all other interventions are considered to be captured by the model calibration (see \S~\ref{sec:calibration}).

The PQ programmatic response is considered to have four key parameters: start date (October 2020), coverage ($c$), G6PD RDT sensitivity ($G$)\todo{though this is actually more like the proportion of the population who are G6PD>0.7 atm, and hence eligible for PQ}, and the effectiveness of PQ in terms of hypnozoite removal ($E$: a combination of efficacy and adherence). The PQ scenario has, from the start date, $(1-c)(1-G)(1-E)$ times the proportion of the population successfully completing treatments not clearing hypnozoites. The baseline scenario has no PQ, and a default value of $0.75$ of the population not clearing hypnozoites on successful treatment completion. To determine if elimination of \pv~is possible by the target date, we consider the case where there is 1.0 coverage, no G6PD deficiency (1.0 of the population are eligible for PQ), and 1.0 effectiveness of PQ, meaning after treatment there are no M15+ with hypnozoites. This is beyond a ``best case scenario'', as these numbers are not achievable. Hence, if \pv~is still present in 2025 in the model predictions, this additional intervention alone will not be sufficient.  

\subsection*{Model calibration} \label{sec:calibration} % and validation 

Data on annual incidence (2011--2018), testing numbers, and demographics were used to calibrate the model for each population stratification and Province (see Additional file~\ref{supp}: Figures~\ref{}--\ref{}). 

The population size was modelled in each Province by group, including transitions from the general population (Gen) to adult males (M15+). The population model (births, deaths, and transitions) was calibrated to fit the known demographics of each Province. I.e. estimated population size, known age and gender breakdown, and expected national lifespan.

The incidence data was divided into 6 clusters by positive \pv~test results in 2018, and a Province was chosen at random from each cluster. Provinces where the borders had changed during 2011--2018 were excluded from being chosen. Except that Pursat was not randomly selected but deliberately chosen, since this is where the trials are currently being conducted. The other five Provinces considered are: Mondul Kiri, Kampong Chhnang, Battambang, Pailin, and Takeo. 

The model was calibrated to the incidence and test data using parameters for the relative susceptibility of the population group to malaria infection; the probability of developing malaria-like symptoms for each person in a given year; the daily probability of testing for people with non-severe malaria-like symptoms such as fever; the daily probability of testing for people with severe malaria-like symptoms; the duration of the latent period (i.e. until hypnozoite reactivation); the proportion incompletely clearing hypnozoites after naturally recovering; and the proportion of new malaria cases that are asymptomatic (see Additional file~\ref{supp}: Table~\ref{calibrated_params}). To allow for changes in the surveillance system in Cambodia (see, for example, \cite{Pengby}), and changes in the other interventions through time, we calibrated to a ``best case'' and ``worst case'' baseline incidence scenario. The ``worst case'' is for the true incidence to be around the higher values of the data and increasing, and the ``best case'' is for the true incidence to be around the lower values of the data and decreasing. The true values are likely inbetween, but this enables us to capture the extreme ends. Uncertainty bounds on the model prediction were generated by sampling $\pm10\%$ of the calibrated parameter values for 30 iterations, and represent 75\% of the subsequent range.

 

%\subsection*{Sensitivity analysis} % maybe not? uncertainty band, sure. But not in reported results for prediction of elimination?

\subsection*{Surveillance}
Given the expected substantial impact on transmission from providing PQ to M15+, we use a negative binomial to estimate how many tests would need to be conducted to be 0.99 confident of detecting at least one case, assuming 100\% sensitivity and specificity of the tests. We also use a binomial to identify the probability of detecting at least one case (assuming 100\% sensitivity and specificity) as the number of tests conducted changes, for the predicted incidence of \pv~in 2020 and 2025. The difference in particular is then indicative of considerations that will need to be given to existing surveillance to be confident that elimination has been reached.

%Results
%This should include the findings of the study including, if appropriate, results of statistical analysis which must be included either in the text or as tables and figures.
\section*{Results}

\subsection*{Current burden of disease in Cambodia}

\subsection*{Model calibration and validation}

\subsection*{Primaquine impact on burden of disease in Cambodia}

%Discussion
%This section should discuss the implications of the findings in context of existing research and highlight limitations of the study.
\section*{Discussion}

%Conclusions
%This should state clearly the main conclusions and provide an explanation of the importance and relevance of the study reported.
\section*{Conclusions}

%List of abbreviations
%If abbreviations are used in the text they should be defined in the text at first use, and a list of abbreviations should be provided.
\section*{List of abbreviations}
Cambodia National Center for Parasitology, Entomology and Malaria Control (CNM)\\
CQ = Chloroquine \\
\pv = \textit{Plasmodium vivax} \\
\pf = \textit{Plasmodium falciparum} \\
PQ = Primaquine \\


%%%%%%%%%%%%%%%%%%%%%%%%%%%%%%%%%%%%%%%%%%%%%%
%%                                          %%
%% Backmatter begins here                   %%
%%                                          %%
%%%%%%%%%%%%%%%%%%%%%%%%%%%%%%%%%%%%%%%%%%%%%%

\begin{backmatter}

\section*{Competing interests}
  The authors declare that they have no competing interests.

\section*{Author's contributions}
    PN, RIH, RMH, AD, DJP and JMM conceived of the project and oversaw the design. PN and RIH curated the data. RMH and RIH developed the transmission model and code implementation, and calibrated the model. RIH, DJP, JMM wrote the surveillance decision support model. RIH, RMH, DJP, AD, JAS, FJIF, JMM, PN prepared the manuscript. All authors read and approved the final manuscript.

\section*{Acknowledgements}
  Text for this section \ldots
%%%%%%%%%%%%%%%%%%%%%%%%%%%%%%%%%%%%%%%%%%%%%%%%%%%%%%%%%%%%%
%%                  The Bibliography                       %%
%%                                                         %%
%%  Bmc_mathpys.bst  will be used to                       %%
%%  create a .BBL file for submission.                     %%
%%  After submission of the .TEX file,                     %%
%%  you will be prompted to submit your .BBL file.         %%
%%                                                         %%
%%                                                         %%
%%  Note that the displayed Bibliography will not          %%
%%  necessarily be rendered by Latex exactly as specified  %%
%%  in the online Instructions for Authors.                %%
%%                                                         %%
%%%%%%%%%%%%%%%%%%%%%%%%%%%%%%%%%%%%%%%%%%%%%%%%%%%%%%%%%%%%%

% if your bibliography is in bibtex format, use those commands:
\bibliographystyle{bmc-mathphys} % Style BST file (bmc-mathphys, vancouver, spbasic).
\bibliography{bmc_article}      % Bibliography file (usually '*.bib' )
% for author-year bibliography (bmc-mathphys or spbasic)
% a) write to bib file (bmc-mathphys only)
% @settings{label, options="nameyear"}
% b) uncomment next line
%\nocite{label}

% or include bibliography directly:
% \begin{thebibliography}
% \bibitem{b1}
% \end{thebibliography}

%%%%%%%%%%%%%%%%%%%%%%%%%%%%%%%%%%%
%%                               %%
%% Figures                       %%
%%                               %%
%% NB: this is for captions and  %%
%% Titles. All graphics must be  %%
%% submitted separately and NOT  %%
%% included in the Tex document  %%
%%                               %%
%%%%%%%%%%%%%%%%%%%%%%%%%%%%%%%%%%%

%%
%% Do not use \listoffigures as most will included as separate files

\section*{Figures}
  \begin{figure}[h!]
  \caption{\csentence{Model calibration for Mondul Kiri.}
      Number of malaria cases as a function of time, from 2011 to 2025. A) General population for the high and increasing baseline incidence. B) Males 15 years and older population for the high and increasing baseline incidence. C) General population for the low and decreasing baseline incidence. D) Males 15 years and older population for the low and decreasing baseline incidence. }
      \end{figure}

\begin{figure}[h!]
  \caption{\csentence{Sample figure title.}
      Figure legend text.}
      \end{figure}

%%%%%%%%%%%%%%%%%%%%%%%%%%%%%%%%%%%
%%                               %%
%% Tables                        %%
%%                               %%
%%%%%%%%%%%%%%%%%%%%%%%%%%%%%%%%%%%

%% Use of \listoftables is discouraged.
%%
\section*{Tables}
\begin{table}[h!]
\caption{Surveillance targets for 0.99 probability of detecting at least one case of \pv in a Province, given the scenarios outlined in \S~\ref{sec:methods}, assuming 100\% sensitivity and specificity of the tests (so a lower bound on number of targets).}
      \begin{tabular}{|l|l|l|l|l|l|l|l|l|l|}
       \hline 
       \multicolumn{2}{|l|}{Year} & \multicolumn{4}{|c|}{2020} & \multicolumn{4}{|c|}{2025} \\ \hline
       \multirow{2}{*}{Scenario} & Incidence & Low & Low & High & High & Low & Low & High & High \\ %\hline 
                                 & Primaquine & None & M 15+ & None & M 15+ & None & M 15+ & None & M 15+ \\ \hline
    \multirow{6}{*}{Province} & Pursat & 441 & 444 & 2,345 & 76 & 76 & 557 & 2,610 & 70 \\ %\hline
                              & Mondul Kiri & 172 & 173 & 48 & 48 & 280 & 1,388 & 54 & 263 \\ %\hline
                              & Kampong Chhnang & 3,798 & 3,819 & 649 & 653 & 5,564 & 26,094 & 614 & 2,998 \\ %\hline 
                              & Battambang & 2,962 & 2,978 & 433 & 436 & 3,916 & 19,191 & 384 & 1,922 \\ %\hline
                              & Pailin & 850 & 855 & 123 & 124 & 1,040 & 4,960 & 122 & 579 \\ %\hline 
                              & Takeo & 14,335 & 14,415 & 2,345 & 2,358 & 18,905 & 89,418 & 2,205 & 10,919 \\ \hline 
      \end{tabular}
\end{table}

%%%%%%%%%%%%%%%%%%%%%%%%%%%%%%%%%%%
%%                               %%
%% Additional Files              %%
%%                               %%
%%%%%%%%%%%%%%%%%%%%%%%%%%%%%%%%%%%

\section*{Additional Files}
  \subsection*{Additional file 1 --- Sample additional file title}
    Additional file descriptions text (including details of how to
    view the file, if it is in a non-standard format or the file extension).  This might
    refer to a multi-page table or a figure.

  \subsection*{Additional file 2 --- Sample additional file title}
    Additional file descriptions text.


\end{backmatter}
\end{document}
